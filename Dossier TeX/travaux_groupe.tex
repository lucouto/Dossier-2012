\section{Évaluation des travaux de groupe}

\subsection{Révélation de Dieu dans l’histoire : Irénée de Lyon}
\subsubsection*{Le corpus étudié}
{\footnotesize
	\fullcite[III-V]{Lyon:contre_heresies}

	\fullcite[« Irénée »]{Balthasar:1968aa}

	\ fullcite{GIRAUD:2011aa}
}
Dans un contacte directe avec le texte d’Irénée, chaque étudiant était sensé de préparer un exposée, de répondre aux éventuelles questions et finalement de préparer une reprise écrite de toute la séance. Cela nous a permis de s’approprier effectivement du texte d’Irénée pour saisir non seulement le contenu théologique de son oeuvre, mais aussi son style propre.

La rencontre avec la théologie d'Irénée de Lyon dans le \emph{Contre les hérésies} est une rencontre pleine de vie et de fraicheur. Une pensée profondément encrée à la fois dans les Écritures et dans la Tradition.  Dans son projet d'offrir une réponse solide contre la fausse gnose, il commence par une explication détaillée de ces doctrines gnostiques afin de pouvoir les combattre ensuite à travers des « preuves des Écritures » afin d'offrir une « argumentation très complète contre tous les hérétiques ». C'est ainsi qu'Irénée pose les Écritures comme la matrice de sa réflexion théologique. Si aujourd'hui, après tout le développement de la méthode historico-critique, nous ne pouvons pas faire le même type d'interprétation typologique et symbolique que fait Irnénée, nous pouvons néanmoins nous inspirer de sa méthode afin de retrouver la richesse et l'importance d'une théologie fondée sur les Écritures.

Un deuxième aspect de la théologie d'Irénée qui est ressortie, c'est la question de l'intégration de l'aspect historique et temporel de la relation entre Dieu et l'homme. Ainsi Irénée introduit la question de l'accoutumance par exemple ou même celle de la récapitulation. D'une manière assez étonnante Irénée n'hésite pas à affirmer que « le Verbe de Dieu, le Fils unique, était de tout temps présent à l'humanité » et que « depuis le commencement, en effet, le Fils, présent à l'ouvrage par lui modelé, révèle le Père à tous ceux à qui le Père le veut, et quand il le veut, et comme il le veut ». Cela nous permet de trouver chez Irénée une véritable théologie de l'espérance qui peut être capable d'être féconde encore à nos jours. Dans un monde qui a du mal à croire, d'abord même en l'homme, la théologie d'Irénée nous invite à faire confiance non seulement en Dieu, mais aussi en l'homme, dans sa liberté.

Cette vision très positive de l'homme qui présente Irénée a pour conséquence une conception du salut comme l'accomplissement de la Création plus que de rédemption du péché. L'homme est pris en compte avec toutes ses faiblesses, mais aussi avec ses vertus et surtout avec sa liberté pour accueillir ou non ce salut qui vient de Dieu. Ce salut passe par la chair et ainsi l'incarnation du Verbe se laisse entrevoir déjà dans l'Ancien Testament. Voilà l'autre élément que nous trouvons comme une opportunité d'actualisation de la théologie de l'évêque de Lyon.  

Finalement, nous gardons toutefois deux résistances par rapport à la lecture du \emph{Contre les hérésies}. D'abord son rapport au judaïsme qui nous semble assez dur et difficilement acceptable dans le contexte de dialogue que nous avons aujourd'hui. Ensuite la lecture assez littérale de l'Apocalypse, même s'il y a la richesse de voir dans la vie du Christ Ressuscité une certaine temporalité de la résurrection de la chair. Le millénarisme développé par Irénée nous semble excessivement attaché à des éléments du texte qui étaient plus liés au contexte historique de l'époque quand ce livre de l'Apocalypse a été rédigé.


\subsection{Bonaventure, \emph{l’Itinéraire de l’esprit vers Dieu}}
\subsubsection*{Le corpus étudié}
{\footnotesize
	\fullcite{Bonaventure:2001aa}
}
\ subsubsection*{L'actualité de la philosophie de Bonaventure}

À la fin du parcours de ce séminaire, la philosophie de S. Bonaventure se présente à moi d'abord comme un appel d'unité. Dans le monde contemporain où l'éclatement du savoir est un fait et la foi est de plus en plus reléguée au domaine du privé, Bonaventure nous montre une voie de possibilité pour une réflexion philosophique déjà dans la foi. Personnellement, je me sens vivement interpelé par ce type de démarche où le chrétien peut témoigner de sa foi au sein même de sa philosophie.

Cependant, une telle approche de la philosophie pose le problème du point de départ. Est-ce que c'est légitime de prendre la Révélation comme principe de la réflexion philosophique ? Une telle philosophie, serait-elle capable d'être acceptée aujourd'hui ? 

L'aspect esthétique de la philosophie de Bonaventure peut nous aider à trouver la voie de dialogue entre cette réflexion philosophique ancrée dans la foi et la société laïcisée dans laquelle nous vivons.  À partir de la contemplation du Beau, mais aussi de l'exercice de l'extase, il est possible de proposer une philosophie qui dépasse la pure réflexion intellectuelle. 

Finalement, je retiens le développement de Bonaventure sur l'âme. Notamment, la manière dont il part de la Trinité pour fonder son argument philosophique de la tripartition de l'âme. C'est ainsi qu'il trouve dans les rapports des facultés de l'âme, la mémoire, l'intelligence et la volonté, une image de la Trinité. 

\subsubsection*{Le rapport entre foi et raison}

Il est évident que pour Bonaventure il n'y a pas de séparation entre raison et foi, même s'il pose une distinction entre les deux. Selon lui, la raison est atteinte par le péché, par conséquent elle a besoin d'être illuminée et réformée par la grâce qui vient du Christ. La foi précède ainsi la raison à tel point que toute réflexion philosophique est faite à l'intérieur de la foi. 

Il s'ensuit que toute connaissance suppose un lien non seulement avec Dieu, mais aussi avec la grâce donnée en Jésus-Christ. La connaissance, même pour la raison naturelle, est une illumination, une participation au Logos éternel. La philosophie, par conséquent, n'a pas d'autonomie par rapport à la foi même si elle garde son domaine propre. 

La question de la connaissance de Dieu, quant à elle, prend une place centrale dans l'\emph{Itinéraire}. Selon Bonaventure, l'esprit peut connaitre Dieu et peut même prouver la nécessité de la Trinité. Cependant, ce mouvement passe par un acte de foi selon la grâce.

En ce qui concerne la théologie, Bonaventure arrive a mettre le Christ en croix au centre de sa réflexion. D'un côté, la raison naturelle, illuminée par la foi, arrive à trouver la nécessité de la Trinité, mais aussi de l'unité de Dieu. De l'autre côté, cette même raison naturelle est incapable de tenir ensemble ces deux affirmations contradictoires sur Dieu. C'est à la croix que Dieu assume toutes nos contradictions et ainsi le Christ en croix est celui qui dépasse les impossibilités de la raison naturelle.

\subsubsection*{Questions pour aller plus loin}

Pour conclure, il y en reste quelques questions qui peuvent permettre d'aller plus loin dans les intuitions d'après la pensée de S. Bonaventure.

Premièrement, la question du rapport entre la foi et la connaissance. Si notre raison est vraiment atteinte par le péché, le non-croyant peut-il arriver à une véritable connaissance du monde sans l'aide de la grâce ? En tant que platonicien, il y a-t-il un parallèle entre cette nécessité d'illumination et la « libération » nécessaire de la caverne chez Platon ?

Deuxièmement, il y a la question du point de départ de la connaissance. Si « \emph{le monde entier s'inscrit dans l'âme}\footnote{Bonaventure, \emph{Les six jours de la création}, Douzième conférence, §16 - Texte supplémentaire de la deuxième séance du séminaire.} », faut-il vraiment le monde sensible ou le retour de l'âme sur elle-même, illuminé par la grâce, suffit ?