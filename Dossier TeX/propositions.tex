% Ce document contiennent les propositions
% 
\newenvironment{references}%
{\begin{hangparas}{.25in}{0}
	\footnotesize
	\textbf{References :\\}
	\footnotesize
}
{\end{hangparas}}


\section{Propositions}

\subsection{Histoire de l’Eglise}

Parmi les thèses développées par les études historiques récentes, on trouve l'idée que l'annonce de l’Évangile chrétien s'est souvent faite à partir du rayonnement que connaissait alors la religion juive (dans les Actes, par exemple, le rapport entre les grande figures apostoliques et les personnages qui sont déjà en relation avec le judaïsme). Ces recherches conduisent à porter l'accent sur les figures de « craignant-Dieu » dans le monde juif au tournant de notre ère ; elles montrent aussi que les questions sur les critères de l'appartenance religieuse (controverses de Paul à Antioche et « concile de Jérusalem ») ne sont pas propres au christianisme, et qu'elles se posaient également au sein d'un judaïsme en expansion.\\

	\begin{references}

		\fullcite{Baslez:2012aa}

	\end{references}

\subsection{Ecclésiologie} 

L’Église est souvent perçue seulement comme une réalité humaine et sociale, or cette approche reviendrait à méconnaître son originalité. Dans la constitution Lumen Gentium, l’Église se comprend comme l’expression et la manifestation du Mystère de Dieu Trinitaire. Pour décrire sa nature qui déborde les réalités humaines, le Concile a utilisé des images complémentaires : l’Église est une convocation du Peuple de Dieu afin d’être rassemblée en un seul corps du Christ, et comme le Temple de l’Esprit au cœur de l’histoire des hommes. \\

\begin{references}

	\fullcite{Comeau:2012aa}

	\fullcite{vatican_LG:1965aa}

\end{references}



\subsection{Patristique : Irénée de Lyon}

Dans son effort pour réfuter « la gnose au nom menteur » des systèmes marcionite et valentinien, Irénée affirme vigoureusement l'unité du Dieu de l'Ancien Testament et de celui qui se révèle en Jésus-Christ ; à travers un commentaire soigneux des Écritures, cherchant à obéir à la « règle de vérité » reçue des Écritures, il affirme la bonté de la Création, et notamment celle de la chair. Cela le conduit à développer une théologie originale de l'histoire, à travers laquelle Dieu, « d'économie » en « économie », accoutume l'homme à vivre en sa présence, et lui donne d'entrer dans le plein exercice de sa liberté.\\

\begin{references}
	\fullcite[III-V]{Lyon:contre_heresies}
	
	\fullcite{Fedou:2012aa}

	\fullcite{GIRAUD:2011aa}
\end{references}


\subsection{Patristique : Le rôle de la grâce dans la propédeutique baptismale} 

Dans le contexte des catéchèses baptismales du IV\ieme siècle de notre ère nous avons des nombreux témoignages, notamment ceux de Basile de Césarée et Jean Chrysostome, sur le type de vie héroïque que les chrétiens étaient appelés à vivre selon la grâce reçue dans leur baptême afin d’être morts au péché, à soi-même et au monde afin de vivre pour le Christ. Mais comment se fait-il que les catéchumènes soient appelés déjà des disciples du Christ dans la propédeutique baptismale ? Si la grâce est déjà à l’œuvre chez les catéchumènes pour les rendre disciples, c’est bien par le baptême qu’ils deviennent des fils de Dieu et des saints intégrés dans l’Église du Christ.\\

\begin{references}
\ fullcite{Schmezer:2012aa}

\ fullcite{Cesaree:1989aa}

\ fullcite{Chrysostome:1957aa}
\end{references}
\subsection{Droit Canon}

Les États libéraux modernes ont la séparation des pouvoirs comme un des éléments constitutifs afin d’assurer la démocratie et la souveraineté du peuple. L’Eglise Catholique Romaine, depuis le XX\ieme~siècle, tente d’intégrer des éléments démocratiques dans son structure de gouvernement afin de défendre à la fois les droits inviolables de la personne humain et le respect de l’égalité fondamentale entre tous les fidèles en vue de la construction du peuple de Dieu dans la vérité et la sainteté. Comment peut-elle concilier cette démarche avec la concentration du pouvoir législatif et exécutif sur la figure du Pontife romain ? Bien que l’objectif ne soit pas d’inscrire dans l’Eglise une séparation des pouvoirs identique à celle de l’Etat, l’Eglise doit s’accommoder d’un accroissement de pouvoir délibératif et de contrôle des assemblées à travers de la collégialité et la synodalité. \\

\begin{references}
	\fullcite{Mestre:1996aa}

	\fullcite{Mestre:2012aa}
\end{references}
\subsection{Théologie fondamentale : La contingence de la Révélation}


L'événement de la Révélation de Dieu en Jésus Christ se manifeste pleinement dans le don de soi à l'autre, un don inconditionnel, absolu et gratuit de sa vie sur la Croix. Cependant, ce don appelle, avant tout, à l'événement de sa réception ; celle-ci, pour sa part, est marquée par la contingence de l'homme et sa liberté. L'Esprit Saint donne à l'homme la grâce d'accéder au don, mais la Révélation n'en assume pas moins le risque de la relation – au plan individuel comme dans le don de la foi fait à l’Église.\\

\begin{references}

	\fullcite{Comeau:2012aa}

	\fullcite[Première étape]{Rahner:2011aa}

\end{references}



\subsection{Médievale : Le rapport entre foi et raison chez S. Bonaventure}

Bonaventure construit l’ensemble de sa théologie à partir de la Révélation et tient ainsi au fait que la raison naturelle est, elle aussi, atteinte par le péché. Comment peut elle accéder à une véritable connaissance du monde ? C’est par la foi en Jésus-Christ que la raison, illuminée et réformée au moyen de la grâce, sera capable d’accéder à cette connaissance véritable du monde et de son Créateur. Reste encore la question de savoir jusqu’à quel point le non-croyant arrive-t-il à connaitre le monde d’une façon véritable.\\

\begin{references}

	\fullcite{Bonaventure:2001aa}

\end{references}

\subsection{Bible : le Canon des Écritures}

L'établissement (ou « clôture ») d'un canon des Écritures, fondateur de l'identité chrétienne, s'inscrit dans le jeu complexe que suppose la compréhension de ces Écritures comme « inspirées » (c'est-à-dire porteuses d'une auto-Révélation de Dieu médiatisée par la parole humaine) : parce que la liste des Écritures inspirées n'est pas elle-même scripturaire, la clôture du canon est à la fois acte d'obéissance et acte d'autorité ; elle donne un statut spécifique, normatif, au témoignage des contemporains de la vie terrestre de Jésus Christ, et ouvre en même temps pour les chrétiens des générations ultérieures un espace légitime de parole et d'interprétation.\\

\begin{references}

	\fullcite{Comeau:2012aa}

	\fullcite{Sesboue:1990aa}

	\fullcite[Le canon des Écritures et sa clôture]{Flichy-Rastoin:2012aa}

\end{references}




\subsection{Bible : Les Actes des Apôtres}

Certains récits des Actes des Apôtres peuvent paraître fort surprenants – tel que la mort d’Ananie et Saphire suite à la fraude. Alors que la première lecture de cet événement souligne la sanction de cette fausse radicalité comme concernant le partage de biens, la lecture plus approfondie pointe l’enjeu de la communion issu de l’Évangile, en opposition à la communion perdu par le péché originel. Cela oriente le lecteur vers l’enjeu théologique des Actes des Apôtres.\\

\begin{references}
	
	\fullcite{Flichy-Rastoin:2012aa}

	\fullcite{Marguerat:2003aa}
	
	Références bibliques : Actes des Apôtres 5,1-11.

\end{references}


\subsection{Christologie}

En Jésus Christ, toutes les figures sont accomplies. Toutefois, cette révélation ne se fait pas simplement sur le mode d'une illumination, mais plutôt dans un clair-obscur, conforme à la manière dont Dieu se donne déjà dans l'Ancien Testament (« Vraiment, tu es un Dieu qui se cache, Dieu d'Israël, sauveur », Is 45, 15). Ce paradoxe qui implique le passage par la Croix, vient pour partie de l'endurcissement du cœur humain ; mais il s'agit aussi, plus profondément, d'une manière pour Dieu de respecter l'homme et son enracinement dans une histoire, et de susciter sa liberté.\\

\begin{references}

	\fullcite{Comeau:2012aa}

	\fullcite{Theobald:2012aa}

	\fullcite[Pensées B 588 et Quatrième lettre à Mlle de Roannez]{Pascal:1972aa}
 
\end{references}


