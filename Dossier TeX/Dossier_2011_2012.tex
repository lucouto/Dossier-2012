%%%%%%%%%%%%%%%%%%%%%%%%%%%%%%%%%%%%%%%%%%%%%%%%%%%%%%%%%%%%%%%%%%%%%%%%%%%%%%%%   
\documentclass[12pt,a4paper]{article} % use larger type; default would be 10pt

%%%%%%%%%%%%%%%%%%%%%%%%%%%%%%%%%%%%%%%%%%%%%%%%%%%%%%%%%%%%
% Este pacote reinicializa o número da nota de rodapé em cada página
\usepackage{perpage} %the perpage package
\MakePerPage{footnote} %the perpage package command

%%%%%%%%%%%%%%%%%%%%%%%%%%%%%%%%%%%%%%%%%%%%%%%%%%%%%%%%%%%%
% Suporte de caracteres especiais acentuados no próprio texto
%%%
\usepackage[utf8]{inputenc}
\usepackage[T1]{fontenc}

%%%%%%%%%%%%%%%%%%%%%%%%%%%%%%%%%%%%%%%%%%%%%%%%%%%%%%%%%%%%
% Pacote para introduzir o comando \lettrine a ser utilisado no início de uma seção
%\usepackage{/Users/lucianoleite/Documents/Chartres/Philosophie/LaTeX_files/malettrine}
\usepackage{lettrine}
%\renewcommand{\LettrineFontHook}{\color[gray]{0.5}}

%%%%%%%%%%%%%%%%%%%%%%%%%%%%%%%%%%%%%%%%%%%%%%%%%%%%%%%%%%%%
%%% Suporte para gerar o documento em Inglês, Português, Grego e Francês.  A última
%%% opção (brazil) é considerado o idioma default do documento.
%%% polutonikogreek é para o grego com os comandos :
%%% \textgreek{} e \selectlanguage{greek} \selectlanguage{english}
\usepackage[polutonikogreek,english,brazil,francais]{babel}
% Duas opções para utilização do grego:
\usepackage[or]{teubner}
%\usepackage{gfsdidot} % este pacote muda a fonte de todo o texto !!!

%%%%%%%%%%%%%%%%%%%%%%%%%%%%%%%%%%%%%%%%%%%%%%%%%%%%%%%%%%%%%%%%%%%%%%%%%%%%%%%%
% Espaçamento entre linhas.
\usepackage{setspace}
%Colocado depois do begin
%\singlespacing % Espaço simples
%\onehalfspacing % Espaço um e meio
%\doublespacing % Espaço duplo 

%%%%%%%%%%%%%%%%%%%%%%%%%%%%%%%%%%%%%%%%%%%%%%%%%%%%%%%%%%%%
%%% dimensões da página
\usepackage[top=3.5cm,left=3.5cm,right=3.5cm,bottom=3.5cm]{geometry} 
\geometry{a4paper} % or letterpaper (US) or a5paper or....

%%% HEADERS & FOOTERS
\usepackage{fancyhdr} % This should be set AFTER setting up the page geometry
\pagestyle{fancy} % options: empty , plain , fancy
\renewcommand{\headrulewidth}{0pt} % customise the layout...
%\lhead{}\chead{}\rhead{}
%\lfoot{}\cfoot{\thepage}\rfoot{}


% Let's get rid of "Section X." in the headers...
\renewcommand{\sectionmark}[1]{\markboth{#1}{}}

\fancyhead [LOE] {}
%\fancyhead [ROE] {Teste}
\renewcommand{\headrulewidth}{0.4pt}
%\renewcommand{\footrulewidth}{0.4pt}


%%%%%%%%%%%%%%%%%%%%%%%%%%%%%%%%%%%%%%%%%%%%%%%%%%%%%%%%%%%%%%%%%%%%%%%%%%%%%%%%
% Definições do pacote BibLatex para nova seção de bibliografia
\usepackage{csquotes}           		% utilisé par biblatex
\usepackage[style=verbose-trad1,
bibencoding=inputenc,			% UTF-8 na bibliografia
language=auto,					% para a lingua
natbib=true,					% comandos "normais" para citação
sortcites=true,
autopunct=true,               			% gérer automatiquement les ponctuations
babel=hyphen,                			 % ajuster les césure pour chaque entrée
hyperref=true,                			% les liens hypertexte, c'est le Bien
block=space,
citetracker=true,]{biblatex}

% Arquivo com definições de bibliografia 
\bibliography{/Volumes/Users/lucianocouto/Documents/Chartres/Philosophie/bibliographie.bib}

\usepackage{hanging}

\begin{document}


\interfootnotelinepenalty=10000 % pour permettre la note en bas de page de rester dans la même page
\begin{titlepage}
	\begin{center}
		\textmd{Premier Cycle de Théologie}
		\vfill
		% Title
		{
		\huge \textbf{Dossier de fin d'année\\
		\vfill
		2011 - 2012}\\[0.4cm]
		
			\vfill
			\vfill
			\large\textbf{Luciano \bsc{Couto}}
		
		
		
			\vfill
			\large\textbf{Tuteur: Etienne \bsc{Vetö}}
			\vfill
			Juin 2012\\
		
			\vfill
		
			Facultés jésuites de Paris\\
			Centre Sèvres\\
		}
		\vfill
	\end{center}
	
	\begin{flushright}
		Première Année de Théologie
	\end{flushright}

\end{titlepage}

%\singlespacing % Espaço simples
\onehalfspacing % Espaço um e meio
%\doublespacing % Espaço duplo 

%Exemple de lettrine :
%\lettrine[lines=2]{B}{ien auparavant} 

\setcounter{tocdepth}{2}
\tableofcontents

\section{Présentation personnelle}

METTRE LA PRÉSENTATION ICI

\subsection*{Liste des cours}
\addcontentsline{toc}{subsection}{Liste des cours}

\subsubsection*{Séminaires}
{\footnotesize
\begin{tabular}{ p{8cm} p{5cm}}
	Bonaventure, l’Itinéraire de l’esprit vers Dieu & Étienne \bsc{Vetö} \\
	
	Révélation de Dieu dans l’histoire : Irénée de Lyon & Michel \bsc{Fédou}
\end{tabular}
}

\subsubsection*{Sessions}
{\footnotesize
\begin{tabular}{ p{8cm} p{5cm} }
	L’église catholique dans le monde : entre unité et diversité & Michel \bsc{Fédou} \newline
	Véronique \bsc{Albanel} \newline
	Agnès \bsc{Kim} \newline
	Marc \bsc{Rastoin} \newline
	\\
	
	Colloque Foi et Raison & M. Emmanuel \bsc{Daublin}\\
	
	
	Génétique : savoirs et pouvoirs, questions éthiques & P. Olivier de \bsc{Dinechin} s.j.\newline
	Dr Delphine \bsc{Héron}
\end{tabular}
}

\subsubsection*{Cours de théologie}
{\footnotesize
\begin{tabular}{ p{8cm} p{5cm} }
	Théologie fondamentale	& Geneviève  \bsc{Comeau}\\
	Atelier : Lecture de textes théologiques & Laure \bsc{Blanchon}\\
	Croire aujourd’hui en Jésus, Christ et Saint de Dieu & Christoph \bsc{Theobald}\\
	Les Pères de l’église grecque au IV\ieme siècle & Catherine \bsc{Schmezer}\\
	Droit canon des religieux & G. \bsc{Ruyssen}\\
	Séquence biblique de 1\iere année & O. \bsc{Flichy}, S. \bsc{Navarro}, M. \bsc{Rastoin}\\
	Une lecture de l’histoire du christianisme II. Périodes moderne et contemporaine & M. \bsc{Hermans}\\
	La christianisation aux quatre premiers siècles & M.-F. \bsc{Baslez}\\
	Les institutions de l’Église & Achille \bsc{Mestre}\\
\end{tabular}
}




\section{Présentation des lectures effectuées}


\begin{hangparas}{.25in}{1}

~\cite{Mestre:1996aa}

 
\end{hangparas}

\subsection{Littérature}
%L'Annonce faite à Marie (extraits) : 
~\cite[extraits]{Claudel:1934aa}

%\end{document}


\subsection*{Dossier de lecture}

\subsubsection*{Histoire des Dogmes - tome I : Le Dieu du salut}

Le choix d’une telle lecture a été fait afin de nous permettre d’avoir une vision panoramique du développement de la dogmatique à partir des grands conciles des premiers siècles de l’Église. Ainsi, en suivant les thèmes de Dieu, la Trinité, le Christ et la sotériologie, nous avons suivi ce parcours ecclésial de la compréhension du mystère du Dieu de Jésus-Christ. Au coeur d’un tel parcours, nous avons trouvé le salut comme une oeuvre opérée par Dieu en faveur de l’homme. Dans le cadre historique choisi par l’auteur, à savoir l’époque patristique, le développement des dogmes est lié aux Symboles de foi qui comportent une structure trinitaire et christologique afin d’accentuer ce salut donné par Dieu et accompli en Jésus-Christ au bénéfice de tout homme. Il est donc clair que le centre du livre n’est pas autre que la question de l’identité de Jésus de Nazareth, appelé Christ et Seigneur, à partir de la compréhension de l’originalité du Dieu révélé en lui et cette oeuvre de salut. 

Mais avant la question du contenu du dogme, un des points qui nous a frappés le plus dans ce travail est le concept de dogme lui-même. L’auteur présente le dogme comme une interprétation des Écritures qui prend en compte les questions et le langage d’une époque et d’une culture données dans une perspective profondément herméneutique. Le concept de dogme est par conséquent intimement lié à celui de tradition, celui-ci comprit comme « le véhicule vivant des affirmations de foi dans la communauté chrétienne ». Cela nous remet au rapport délicat et complexe entre la Tradition et l’Écriture puisque celle-ci demeure le critère décisif de la validité de tout dogme. Ainsi, le dogme n’est pas une invention « arbitraire » de l’Église afin de défendre de manière autoritaire sa manière de penser, mais il est tout simplement « l’acte d’interprétation de la parole de Dieu consignée dans l’Écriture », construit dans l’histoire de la Tradition, et que « se tiendra comme référence pour les assemblées conciliaires ». 

Dans ce sens, l’exemple du dogme de la divinité du Fils attesté par le concile de Nicée est un excellent exemple de comment l’Église veut garder l’authenticité du message de l’Évangile à travers le temps et le dialogue avec les cultures. C’est là que pour la première fois les mots utilisés dans le Symbole de foi, cellule-mère de tout le développement dogmatique dans l’Église, ne viennent pas de l’Écriture, mais de la philosophie grecque. À Nicée nous voyons l’Église qui répond à des questions nouvelles en traduisant une règle de foi dans un langage culturel nouveau. En outre, Nicée nous montre que la prise de conscience de l’autorité conciliaire a été aussi le résultat d’un processus herméneutique. C’est dans la réception du concile qu’une ecclésiologie conciliaire a pu se développer. À partir de la conviction de que l’Eglise universelle est représentée dans le concile, l’Eglise qui reçoit ses fruits prend conscience de l’autorité conciliaire en même temps que ses décisions sont prises dans un caractère définitif et irrévocable. 

Ce défi, selon nous, est toujours actuel et s’impose non seulement à la dogmatique, mais à tout discours de foi de l’Église. Comment pouvons-nous prendre conscience des questions qui traversent notre temps, dans nos sociétés et cultures tellement diverses et en même temps globalisées afin d’y répondre dans un langage actuel et compréhensible capable d’atteindre les gens d’une façon qu’ils se sentent concernés par ce discours ?



\section{Les travaux écrits}

\subsection{L’engagement chrétien à partir de la théologie du \emph{Dieu Crucifié}}
{\footnotesize
\begin{flushright}
 \emph{dirigé par Geneviève \bsc{Comeau}}
 \end{flushright}
 }
Le but principal du travail était d’entrer en contact avec la théologie trinitaire de Moltmann à travers son oeuvre, [i]Le Dieu Crucifié[i], pour ensuite dégager la question de la compréhension de la Trinité dans notre société contemporaine et les conséquences qui en découlent.

Le point de départ était celui de Moltmann : une théologie qui se veut chrétienne doit être pensée à partir de la donnée fondamentale de notre foi, à savoir la Passion et la Résurrection de Jésus de Nazareth, le Christ de Dieu, qui nous fait participer à la vie divine. Nous constatons avec l’auteur que la métaphysique classique, en tant qu’approche théologique, a valorisé au XX\ieme siècle  davantage les concepts philosophiques sur Dieu que les faits de la Révélation. Ainsi, elle est devenue incapable d’expliquer de manière satisfaisante non seulement la Trinité, mais aussi les relations  des personnes  divines avec les hommes et la création.

Or, même si une telle démarche se justifie dans la théologie, elle peut produire une image très impersonnelle de Dieu, comme une réalité lointaine de l’homme et du monde où il vit. L’homme reste ainsi en dehors d’une relation vivante avec le Dieu vivant et vrai.

Nous avons trouvé dans [i]Le Dieu Crucifié[i] des pistes importantes pour dépasser cette limite et trouver ainsi d’autres manières de faire de la théologie, qui puissent contribuer à créer ou intensifier la conscience du fait que le Dieu Créateur  de toutes choses, l’Être absolu, le Premier Principe, s’est révélé en son Fils et nous appelle à rentrer  en relation avec Lui.

Une théologie qui prend en compte l’histoire de la Révélation et la Révélation de Dieu dans l’histoire est capable de développer d’autres aspects de la vie trinitaire à partir  de l’expérience du Dieu Crucifié, afin d’interpeller les hommes de notre temps et de les inviter  s’ouvrir à une relation personnelle avec ce Dieu qui se révèle en Jésus-Christ. Telle est la démarche de Moltmann,  que nous étudierons  dans la première partie.

L’expérience de foi qui se dégage d’une telle démarche est un engagement de l’homme dans le monde dans lequel il vit. Or nous pensons que l’engagement politique proposé par Moltmann était, en effet, très adapté au contexte du XX\ieme siècle. Cependant, il nous semble qu’aujourd’hui nous pouvons trouver  d’autres  manières, plus éloquentes, d’engager l’expérience de foi dans le monde afin de promouvoir cette rencontre  personnelle avec notre Sauveur.

Un tel engagement peut donc prendre différentes formes, sous le mode du don qui me semble être le plus éloquent dans notre monde aujourd’hui. À partir de l’image de Dieu en tant qu’événement d’amour entre les personnes de la Trinité qui se communique à l’homme pour le sauver et le libérer, l’homme est appelé à son tour à entrer dans le même mouvement d’amour et de don, non seulement envers son semblable, mais aussi envers toute la création. Le message chrétien peut ainsi trouver bon accueil même dans les milieux laïcs, mais où se rencontre un esprit de solidarité. C’est ce que montre, par exemple, le grand rôle et l’immense développement du Tiers Secteur (l’économie solidaire) non seulement en Amérique, mais aussi en Europe. Voilà une des voies actuelles possibles de l’engagement chrétien dans le monde, afin de permettre concrètement un travail de justice et de libération selon l’Évangile.  


\section{Évaluation des travaux de groupe}

\subsection{Révélation de Dieu dans l’histoire : Irénée de Lyon}
\subsubsection*{Le corpus étudié}
{\footnotesize
	\fullcite[III-V]{Lyon:contre_heresies}

	\fullcite[« Irénée »]{Balthasar:1968aa}

	\ fullcite{GIRAUD:2011aa}
}
Dans un contacte directe avec le texte d’Irénée, chaque étudiant était sensé de préparer un exposée, de répondre aux éventuelles questions et finalement de préparer une reprise écrite de toute la séance. Cela nous a permis de s’approprier effectivement du texte d’Irénée pour saisir non seulement le contenu théologique de son oeuvre, mais aussi son style propre.

La rencontre avec la théologie d'Irénée de Lyon dans le \emph{Contre les hérésies} est une rencontre pleine de vie et de fraicheur. Une pensée profondément encrée à la fois dans les Écritures et dans la Tradition.  Dans son projet d'offrir une réponse solide contre la fausse gnose, il commence par une explication détaillée de ces doctrines gnostiques afin de pouvoir les combattre ensuite à travers des « preuves des Écritures » afin d'offrir une « argumentation très complète contre tous les hérétiques ». C'est ainsi qu'Irénée pose les Écritures comme la matrice de sa réflexion théologique. Si aujourd'hui, après tout le développement de la méthode historico-critique, nous ne pouvons pas faire le même type d'interprétation typologique et symbolique que fait Irnénée, nous pouvons néanmoins nous inspirer de sa méthode afin de retrouver la richesse et l'importance d'une théologie fondée sur les Écritures.

Un deuxième aspect de la théologie d'Irénée qui est ressortie, c'est la question de l'intégration de l'aspect historique et temporel de la relation entre Dieu et l'homme. Ainsi Irénée introduit la question de l'accoutumance par exemple ou même celle de la récapitulation. D'une manière assez étonnante Irénée n'hésite pas à affirmer que « le Verbe de Dieu, le Fils unique, était de tout temps présent à l'humanité » et que « depuis le commencement, en effet, le Fils, présent à l'ouvrage par lui modelé, révèle le Père à tous ceux à qui le Père le veut, et quand il le veut, et comme il le veut ». Cela nous permet de trouver chez Irénée une véritable théologie de l'espérance qui peut être capable d'être féconde encore à nos jours. Dans un monde qui a du mal à croire, d'abord même en l'homme, la théologie d'Irénée nous invite à faire confiance non seulement en Dieu, mais aussi en l'homme, dans sa liberté.

Cette vision très positive de l'homme qui présente Irénée a pour conséquence une conception du salut comme l'accomplissement de la Création plus que de rédemption du péché. L'homme est pris en compte avec toutes ses faiblesses, mais aussi avec ses vertus et surtout avec sa liberté pour accueillir ou non ce salut qui vient de Dieu. Ce salut passe par la chair et ainsi l'incarnation du Verbe se laisse entrevoir déjà dans l'Ancien Testament. Voilà l'autre élément que nous trouvons comme une opportunité d'actualisation de la théologie de l'évêque de Lyon.  

Finalement, nous gardons toutefois deux résistances par rapport à la lecture du \emph{Contre les hérésies}. D'abord son rapport au judaïsme qui nous semble assez dur et difficilement acceptable dans le contexte de dialogue que nous avons aujourd'hui. Ensuite la lecture assez littérale de l'Apocalypse, même s'il y a la richesse de voir dans la vie du Christ Ressuscité une certaine temporalité de la résurrection de la chair. Le millénarisme développé par Irénée nous semble excessivement attaché à des éléments du texte qui étaient plus liés au contexte historique de l'époque quand ce livre de l'Apocalypse a été rédigé.


\subsection{Bonaventure, \emph{l’Itinéraire de l’esprit vers Dieu}}
\subsubsection*{Le corpus étudié}
{\footnotesize
	\fullcite{Bonaventure:2001aa}
}
\ subsubsection*{L'actualité de la philosophie de Bonaventure}

À la fin du parcours de ce séminaire, la philosophie de S. Bonaventure se présente à moi d'abord comme un appel d'unité. Dans le monde contemporain où l'éclatement du savoir est un fait et la foi est de plus en plus reléguée au domaine du privé, Bonaventure nous montre une voie de possibilité pour une réflexion philosophique déjà dans la foi. Personnellement, je me sens vivement interpelé par ce type de démarche où le chrétien peut témoigner de sa foi au sein même de sa philosophie.

Cependant, une telle approche de la philosophie pose le problème du point de départ. Est-ce que c'est légitime de prendre la Révélation comme principe de la réflexion philosophique ? Une telle philosophie, serait-elle capable d'être acceptée aujourd'hui ? 

L'aspect esthétique de la philosophie de Bonaventure peut nous aider à trouver la voie de dialogue entre cette réflexion philosophique ancrée dans la foi et la société laïcisée dans laquelle nous vivons.  À partir de la contemplation du Beau, mais aussi de l'exercice de l'extase, il est possible de proposer une philosophie qui dépasse la pure réflexion intellectuelle. 

Finalement, je retiens le développement de Bonaventure sur l'âme. Notamment, la manière dont il part de la Trinité pour fonder son argument philosophique de la tripartition de l'âme. C'est ainsi qu'il trouve dans les rapports des facultés de l'âme, la mémoire, l'intelligence et la volonté, une image de la Trinité. 

\subsubsection*{Le rapport entre foi et raison}

Il est évident que pour Bonaventure il n'y a pas de séparation entre raison et foi, même s'il pose une distinction entre les deux. Selon lui, la raison est atteinte par le péché, par conséquent elle a besoin d'être illuminée et réformée par la grâce qui vient du Christ. La foi précède ainsi la raison à tel point que toute réflexion philosophique est faite à l'intérieur de la foi. 

Il s'ensuit que toute connaissance suppose un lien non seulement avec Dieu, mais aussi avec la grâce donnée en Jésus-Christ. La connaissance, même pour la raison naturelle, est une illumination, une participation au Logos éternel. La philosophie, par conséquent, n'a pas d'autonomie par rapport à la foi même si elle garde son domaine propre. 

La question de la connaissance de Dieu, quant à elle, prend une place centrale dans l'\emph{Itinéraire}. Selon Bonaventure, l'esprit peut connaitre Dieu et peut même prouver la nécessité de la Trinité. Cependant, ce mouvement passe par un acte de foi selon la grâce.

En ce qui concerne la théologie, Bonaventure arrive a mettre le Christ en croix au centre de sa réflexion. D'un côté, la raison naturelle, illuminée par la foi, arrive à trouver la nécessité de la Trinité, mais aussi de l'unité de Dieu. De l'autre côté, cette même raison naturelle est incapable de tenir ensemble ces deux affirmations contradictoires sur Dieu. C'est à la croix que Dieu assume toutes nos contradictions et ainsi le Christ en croix est celui qui dépasse les impossibilités de la raison naturelle.

\subsubsection*{Questions pour aller plus loin}

Pour conclure, il y en reste quelques questions qui peuvent permettre d'aller plus loin dans les intuitions d'après la pensée de S. Bonaventure.

Premièrement, la question du rapport entre la foi et la connaissance. Si notre raison est vraiment atteinte par le péché, le non-croyant peut-il arriver à une véritable connaissance du monde sans l'aide de la grâce ? En tant que platonicien, il y a-t-il un parallèle entre cette nécessité d'illumination et la « libération » nécessaire de la caverne chez Platon ?

Deuxièmement, il y a la question du point de départ de la connaissance. Si « \emph{le monde entier s'inscrit dans l'âme}\footnote{Bonaventure, \emph{Les six jours de la création}, Douzième conférence, §16 - Texte supplémentaire de la deuxième séance du séminaire.} », faut-il vraiment le monde sensible ou le retour de l'âme sur elle-même, illuminé par la grâce, suffit ?

\section{Présentation des autres activités}
Parmi les autres activités développées au cours de l'année, je peux souligner le service dans les sessions de la Communauté du Chemin Neuf pour les jeunes entre 16 et 18 ans. Au cours des trois week-ends, les jeunes se sont rencontrés pour travailler les thèmes de la famille, de l'Esprit Saint et de leurs amitiés. Plus personnellement mon rôle était d’accompagner les frères et soeurs de la communauté dans l’organisation et pendants les rencontres de faire des enseignements liés aux thèmes. 

 Toujours avec la communauté, j'ai pris la charge de préparation de la Journée Mondiale de la Jeunesse (JMJ) qui aura lieu en 2013 au Brésil. En vue de cette mission que m’a été confiée, j’était au Brésil pendant les vacances de printemps pour rencontrer les responsables nationaux de la JMJ 2013 à Rio afin de présenter le projet de la Communauté du Chemin Neuf. De même, nous avons organisé un Festival avec une cinquantaine des jeunes qui se mettent dans l’organisation avec nous.
 
 


% Ce document contiennent les propositions
% 
\newenvironment{references}%
{\begin{hangparas}{.25in}{0}
	\footnotesize
	\textbf{References :\\}
	\footnotesize
}
{\end{hangparas}}


\section{Propositions}

\subsection{Histoire de l’Eglise}

Parmi les thèses développées par les études historiques récentes, on trouve l'idée que l'annonce de l’Évangile chrétien s'est souvent faite à partir du rayonnement que connaissait alors la religion juive (dans les Actes, par exemple, le rapport entre les grande figures apostoliques et les personnages qui sont déjà en relation avec le judaïsme). Ces recherches conduisent à porter l'accent sur les figures de « craignant-Dieu » dans le monde juif au tournant de notre ère ; elles montrent aussi que les questions sur les critères de l'appartenance religieuse (controverses de Paul à Antioche et « concile de Jérusalem ») ne sont pas propres au christianisme, et qu'elles se posaient également au sein d'un judaïsme en expansion.\\

	\begin{references}

		\fullcite{Baslez:2012aa}

	\end{references}

\subsection{Ecclésiologie} 

L’Église est souvent perçue seulement comme une réalité humaine et sociale, or cette approche reviendrait à méconnaître son originalité. Dans la constitution Lumen Gentium, l’Église se comprend comme l’expression et la manifestation du Mystère de Dieu Trinitaire. Pour décrire sa nature qui déborde les réalités humaines, le Concile a utilisé des images complémentaires : l’Église est une convocation du Peuple de Dieu afin d’être rassemblée en un seul corps du Christ, et comme le Temple de l’Esprit au cœur de l’histoire des hommes. \\

\begin{references}

	\fullcite{Comeau:2012aa}

	\fullcite{vatican_LG:1965aa}

\end{references}



\subsection{Patristique : Irénée de Lyon}

Dans son effort pour réfuter « la gnose au nom menteur » des systèmes marcionite et valentinien, Irénée affirme vigoureusement l'unité du Dieu de l'Ancien Testament et de celui qui se révèle en Jésus-Christ ; à travers un commentaire soigneux des Écritures, cherchant à obéir à la « règle de vérité » reçue des Écritures, il affirme la bonté de la Création, et notamment celle de la chair. Cela le conduit à développer une théologie originale de l'histoire, à travers laquelle Dieu, « d'économie » en « économie », accoutume l'homme à vivre en sa présence, et lui donne d'entrer dans le plein exercice de sa liberté.\\

\begin{references}
	\fullcite[III-V]{Lyon:contre_heresies}
	
	\fullcite{Fedou:2012aa}

	\fullcite{GIRAUD:2011aa}
\end{references}


\subsection{Patristique : Le rôle de la grâce dans la propédeutique baptismale} 

Dans le contexte des catéchèses baptismales du IV\ieme siècle de notre ère nous avons des nombreux témoignages, notamment ceux de Basile de Césarée et Jean Chrysostome, sur le type de vie héroïque que les chrétiens étaient appelés à vivre selon la grâce reçue dans leur baptême afin d’être morts au péché, à soi-même et au monde afin de vivre pour le Christ. Mais comment se fait-il que les catéchumènes soient appelés déjà des disciples du Christ dans la propédeutique baptismale ? Si la grâce est déjà à l’œuvre chez les catéchumènes pour les rendre disciples, c’est bien par le baptême qu’ils deviennent des fils de Dieu et des saints intégrés dans l’Église du Christ.\\

\begin{references}
\ fullcite{Schmezer:2012aa}

\ fullcite{Cesaree:1989aa}

\ fullcite{Chrysostome:1957aa}
\end{references}
\subsection{Droit Canon}

Les États libéraux modernes ont la séparation des pouvoirs comme un des éléments constitutifs afin d’assurer la démocratie et la souveraineté du peuple. L’Eglise Catholique Romaine, depuis le XX\ieme~siècle, tente d’intégrer des éléments démocratiques dans son structure de gouvernement afin de défendre à la fois les droits inviolables de la personne humain et le respect de l’égalité fondamentale entre tous les fidèles en vue de la construction du peuple de Dieu dans la vérité et la sainteté. Comment peut-elle concilier cette démarche avec la concentration du pouvoir législatif et exécutif sur la figure du Pontife romain ? Bien que l’objectif ne soit pas d’inscrire dans l’Eglise une séparation des pouvoirs identique à celle de l’Etat, l’Eglise doit s’accommoder d’un accroissement de pouvoir délibératif et de contrôle des assemblées à travers de la collégialité et la synodalité. \\

\begin{references}
	\fullcite{Mestre:1996aa}

	\fullcite{Mestre:2012aa}
\end{references}
\subsection{Théologie fondamentale : La contingence de la Révélation}


L'événement de la Révélation de Dieu en Jésus Christ se manifeste pleinement dans le don de soi à l'autre, un don inconditionnel, absolu et gratuit de sa vie sur la Croix. Cependant, ce don appelle, avant tout, à l'événement de sa réception ; celle-ci, pour sa part, est marquée par la contingence de l'homme et sa liberté. L'Esprit Saint donne à l'homme la grâce d'accéder au don, mais la Révélation n'en assume pas moins le risque de la relation – au plan individuel comme dans le don de la foi fait à l’Église.\\

\begin{references}

	\fullcite{Comeau:2012aa}

	\fullcite[Première étape]{Rahner:2011aa}

\end{references}



\subsection{Médievale : Le rapport entre foi et raison chez S. Bonaventure}

Bonaventure construit l’ensemble de sa théologie à partir de la Révélation et tient ainsi au fait que la raison naturelle est, elle aussi, atteinte par le péché. Comment peut elle accéder à une véritable connaissance du monde ? C’est par la foi en Jésus-Christ que la raison, illuminée et réformée au moyen de la grâce, sera capable d’accéder à cette connaissance véritable du monde et de son Créateur. Reste encore la question de savoir jusqu’à quel point le non-croyant arrive-t-il à connaitre le monde d’une façon véritable.\\

\begin{references}

	\fullcite{Bonaventure:2001aa}

\end{references}

\subsection{Bible : le Canon des Écritures}

L'établissement (ou « clôture ») d'un canon des Écritures, fondateur de l'identité chrétienne, s'inscrit dans le jeu complexe que suppose la compréhension de ces Écritures comme « inspirées » (c'est-à-dire porteuses d'une auto-Révélation de Dieu médiatisée par la parole humaine) : parce que la liste des Écritures inspirées n'est pas elle-même scripturaire, la clôture du canon est à la fois acte d'obéissance et acte d'autorité ; elle donne un statut spécifique, normatif, au témoignage des contemporains de la vie terrestre de Jésus Christ, et ouvre en même temps pour les chrétiens des générations ultérieures un espace légitime de parole et d'interprétation.\\

\begin{references}

	\fullcite{Comeau:2012aa}

	\fullcite{Sesboue:1990aa}

	\fullcite[Le canon des Écritures et sa clôture]{Flichy-Rastoin:2012aa}

\end{references}




\subsection{Bible : Les Actes des Apôtres}

Certains récits des Actes des Apôtres peuvent paraître fort surprenants – tel que la mort d’Ananie et Saphire suite à la fraude. Alors que la première lecture de cet événement souligne la sanction de cette fausse radicalité comme concernant le partage de biens, la lecture plus approfondie pointe l’enjeu de la communion issu de l’Évangile, en opposition à la communion perdu par le péché originel. Cela oriente le lecteur vers l’enjeu théologique des Actes des Apôtres.\\

\begin{references}
	
	\fullcite{Flichy-Rastoin:2012aa}

	\fullcite{Marguerat:2003aa}
	
	Références bibliques : Actes des Apôtres 5,1-11.

\end{references}


\subsection{Christologie}

En Jésus Christ, toutes les figures sont accomplies. Toutefois, cette révélation ne se fait pas simplement sur le mode d'une illumination, mais plutôt dans un clair-obscur, conforme à la manière dont Dieu se donne déjà dans l'Ancien Testament (« Vraiment, tu es un Dieu qui se cache, Dieu d'Israël, sauveur », Is 45, 15). Ce paradoxe qui implique le passage par la Croix, vient pour partie de l'endurcissement du cœur humain ; mais il s'agit aussi, plus profondément, d'une manière pour Dieu de respecter l'homme et son enracinement dans une histoire, et de susciter sa liberté.\\

\begin{references}

	\fullcite{Comeau:2012aa}

	\fullcite{Theobald:2012aa}

	\fullcite[Pensées B 588 et Quatrième lettre à Mlle de Roannez]{Pascal:1972aa}
 
\end{references}




%\include{bilan}

%\include{prospective}


\end{document}
