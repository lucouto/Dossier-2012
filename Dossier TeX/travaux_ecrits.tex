\section{Les travaux écrits}

\subsection{L’engagement chrétien à partir de la théologie du \emph{Dieu Crucifié}}
{\footnotesize
\begin{flushright}
 \emph{dirigé par Geneviève \bsc{Comeau}}
 \end{flushright}
 }
Le but principal du travail était d’entrer en contact avec la théologie trinitaire de Moltmann à travers son oeuvre, [i]Le Dieu Crucifié[i], pour ensuite dégager la question de la compréhension de la Trinité dans notre société contemporaine et les conséquences qui en découlent.

Le point de départ était celui de Moltmann : une théologie qui se veut chrétienne doit être pensée à partir de la donnée fondamentale de notre foi, à savoir la Passion et la Résurrection de Jésus de Nazareth, le Christ de Dieu, qui nous fait participer à la vie divine. Nous constatons avec l’auteur que la métaphysique classique, en tant qu’approche théologique, a valorisé au XX\ieme siècle  davantage les concepts philosophiques sur Dieu que les faits de la Révélation. Ainsi, elle est devenue incapable d’expliquer de manière satisfaisante non seulement la Trinité, mais aussi les relations  des personnes  divines avec les hommes et la création.

Or, même si une telle démarche se justifie dans la théologie, elle peut produire une image très impersonnelle de Dieu, comme une réalité lointaine de l’homme et du monde où il vit. L’homme reste ainsi en dehors d’une relation vivante avec le Dieu vivant et vrai.

Nous avons trouvé dans [i]Le Dieu Crucifié[i] des pistes importantes pour dépasser cette limite et trouver ainsi d’autres manières de faire de la théologie, qui puissent contribuer à créer ou intensifier la conscience du fait que le Dieu Créateur  de toutes choses, l’Être absolu, le Premier Principe, s’est révélé en son Fils et nous appelle à rentrer  en relation avec Lui.

Une théologie qui prend en compte l’histoire de la Révélation et la Révélation de Dieu dans l’histoire est capable de développer d’autres aspects de la vie trinitaire à partir  de l’expérience du Dieu Crucifié, afin d’interpeller les hommes de notre temps et de les inviter  s’ouvrir à une relation personnelle avec ce Dieu qui se révèle en Jésus-Christ. Telle est la démarche de Moltmann,  que nous étudierons  dans la première partie.

L’expérience de foi qui se dégage d’une telle démarche est un engagement de l’homme dans le monde dans lequel il vit. Or nous pensons que l’engagement politique proposé par Moltmann était, en effet, très adapté au contexte du XX\ieme siècle. Cependant, il nous semble qu’aujourd’hui nous pouvons trouver  d’autres  manières, plus éloquentes, d’engager l’expérience de foi dans le monde afin de promouvoir cette rencontre  personnelle avec notre Sauveur.

Un tel engagement peut donc prendre différentes formes, sous le mode du don qui me semble être le plus éloquent dans notre monde aujourd’hui. À partir de l’image de Dieu en tant qu’événement d’amour entre les personnes de la Trinité qui se communique à l’homme pour le sauver et le libérer, l’homme est appelé à son tour à entrer dans le même mouvement d’amour et de don, non seulement envers son semblable, mais aussi envers toute la création. Le message chrétien peut ainsi trouver bon accueil même dans les milieux laïcs, mais où se rencontre un esprit de solidarité. C’est ce que montre, par exemple, le grand rôle et l’immense développement du Tiers Secteur (l’économie solidaire) non seulement en Amérique, mais aussi en Europe. Voilà une des voies actuelles possibles de l’engagement chrétien dans le monde, afin de permettre concrètement un travail de justice et de libération selon l’Évangile.  
